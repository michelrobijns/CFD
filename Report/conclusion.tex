We have presented a novel numerical method for solving the incompressible Navier-Stokes equations. This novel method is proposed as an alternative to established numerical methods such as the finite-difference method, finite-volume method, and finite-element method. The method was used to solve the incompressible Navier-Stokes equations on a two-dimensional lid-driven cavity, a classical test problem for the validation of Navier-Stokes codes. After comparing its results (i.e., the stream function, vorticity, and static pressure) with benchmark solutions, we can conclude that the novel method works and produces a correct solution.

To capture the geometric structure of the governing equations, we defined its physical quantities through integral values over the elements of the mesh. Depending on whether a given physical quantity was a point, line, or area density, its corresponding discrete representation “lived” at the associated zero, one, or two dimensional mesh elements. This is both an elegant and a natural approach and its advantages are manyfold. Most of the method is mathematically exact because the physical quantities are defined through integral values. Numerical errors are introduced when switching between the inner and outer-oriented grids, when computing the convection term, and when advancing the solution in time. The present method also results in highly sparse matrices that can be used to write efficient code when writing 3D solvers that are inherently computationally expensive.

Future implementations of the present method should explore improving the approximation of the convective term as the poor approximation of the convective term turns out to be the bottleneck. Once that bottleneck is out of the way, it makes sense to incorporate multithreaded sparse matrix operations and higher-order time-stepping methods to improve memory use, execution time, and the rate of convergence.