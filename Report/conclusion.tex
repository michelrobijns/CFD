We have presented a novel numerical method for solving the incompressible Navier-Stokes equations. The present method is proposed as an alternative to conventional numerical methods such as the finite-difference method, finite-volume method, and finite-element method. The present method was used to solve the incompressible Navier-Stokes equations on a two-dimensional lid-driven cavity, a classical test problem for the validation of Navier-Stokes codes. After comparing its results (i.e., the stream function, vorticity, and static pressure) with benchmark results in \parencite{botella1998benchmark}, it can be concluded that the novel method works and produces correct results.

To capture the geometric structure of the governing equations, we defined its physical quantities through integral values over the elements of the mesh. Depending on whether a given physical quantity was a point, line, or area density, its corresponding discrete representation “lived” at the associated zero, one, or two dimensional mesh elements. This is both an elegant and a natural approach and its advantages are manyfold. Numerical errors are only introduced when switching between the inner and outer-oriented grids, when computing the convection term, and when advancing the solution in time. The identification of error sources is trivial as the introduction of error is limited to only a small number of stages of the solution process. The present method results in highly sparse matrices that can be exploited to write computationally efficient code.

Future implementations of the present method should explore improved approximations of the convective term as the poor approximation of the convective term turns out to be the bottleneck of accuracy. Once the poor approximation of the convective term is improved, it may be benefition to incorporate multithreaded sparse matrix operations and higher-order time-stepping methods to improve execution time, memory use, and the rate of convergence. Aspiring developers of DEC based codes are also strongly advised to consider the method described in \parencite{elcott2005discrete}, whose authors use an implicit time-stepping scheme that is inherently independent of the timestep $\Delta t$.