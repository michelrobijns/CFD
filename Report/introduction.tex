The vast majority of conventional numerical methods for discretizing and solving partial differential equations (for instance, the finite difference method, the finite volume method, and the finite element method) are based on the idea of approximating derivatives as a means to approximate partial differential equations. The discrete derivative either requires a small spacing between the nodal points or computationally expensive higher-order approximations to yield a small discretization error. The idea is valid, but has a number of disadvantages; truncation errors are introduced right from the start and the magnitude of the truncation error is a direct function of the spacing between the nodal points.

The objective of this report is the presentation of a novel method for solving the incompressible Navier-Stokes equations. The main aim of this report is to demonstrate that the method works and to discuss its advantages over conventional numerical methods. The method follows a geometric approach that preserves the physical relations that are inherently geometrical. This remarkable property is achieved by using tools from a vast mathematical field called discrete exterior calculus (DEC). A key ingredient in this approach is avoiding approximations of derivatives entirely by expressing all state variables as integrated quantities that "live" on associated the mesh elements. The method is therefore mathematically exact until one the last stages of the process where the spacing between the nodal points is brought into the equation.

The present method will be used to solve the incompressible Navier-Stokes equations on a two-dimensional lid-driven cavity. The lid-driven cavity flow problem is a classical test problem for the validation of Navier-Stokes codes and there is a plethora of both experimental and numerical results that can be used for the purpose of validation. In Chapter \ref{cha:physics} we will derive the governing equations and formally present the lid-driven cavity flow problem. In Chapter \ref{cha:mathematics} we will introduce the numerical framework by example and in a step-by-step manner. In Chapter \ref{cha:code} we will show the most important excerpt of the program code and look for optimizations of the code in terms of execution time and memory use. Finally, in Chapter \ref{cha:results} we will validate the results by means of a comparison with benchmark results from a 1997 paper by O.~Botella and R.~Peyret \parencite{botella1998benchmark}.

% Checked 4/24/16 7:30 PM
