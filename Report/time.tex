\section{Time Marching}

The Navier-Stokes equations were rewritten in terms of incidence matrices and Hodge matrices, as follows:
\begin{align}
    \stepcounter{equation}
    \tag{{\theequation}a}
    \label{eq:timeContinuity}
    &\tilde{\mathbb{E}}^{(2,1)} \mathbb{H}^{(\tilde{1},1)} \mathbf{u}  + \mathbf{u}^{}_{\text{norm}} = 0 \vphantom{\frac{\partial^{(1)}}{\partial}} \\
    \tag{{\theequation}b}
    \label{eq:timeVorticity}
    &\xi = \mathbb{E}^{(2,1)} \mathbf{u} \vphantom{\frac{\partial^{(1)}}{\partial}} \\
    \tag{{\theequation}c}
    \label{eq:timeMomentum}
    &\frac{\partial \mathbf{u}}{\partial t} + \text{convective} + \mathbb{E}^{(1,0)} P + \frac{1}{\text{Re}} \mathbb{H}^{(1,\tilde{1})} \tilde{\mathbb{E}}^{(1,0)} \mathbb{H}^{(\tilde{0},2)} \mathbb{E}^{(2,1)} \mathbf{u} + \frac{1}{\text{Re}} \mathbf{u}_{\text{pres}} = 0
\end{align}

We will be using the forward Euler method, an explicit method, to advance the solution in time. Suppose we have a first order differential equation given by
\begin{equation}
    \label{eq:Euler1}
    \frac{d \mathbf{u}}{dt} + f(\mathbf{u}) = 0
\end{equation}
Replacing the time derivative by the forward Euler scheme, we have
\begin{equation}
    \label{eq:Euler2}
    \frac{\mathbf{u}^{n+1} - \mathbf{u}^{n}}{\Delta t} + f(\mathbf{u}^{n}) = 0
\end{equation}
where $n$ denotes a certain point in time and $\Delta t$ denotes the timestep. Equation \eqref{eq:Euler2} can be rearranged such that the solution at time $n+1$ is expressed as a function of the solution at time $n$:
\begin{equation}
    \label{eq:time1}
    \mathbf{u}^{n+1} = \mathbf{u}^{n} - \Delta t \; f(\mathbf{u}^{n})
\end{equation}
This method is a first-order method because it produces an error of $O(\Delta t)$. Time-stepping using the forward Euler method is as simple as it gets, but its drawback is that it requires extremely small values of $\Delta t$ to be numerially stable. In spite of this major disadvantage, let us go ahead and discretize the time derivative in Equation \eqref{eq:timeMomentum} using a forward Euler scheme. It will be trivial to replace the time-stepping method by a higher-order method at a later stage. Replacing the time derivative in Equation \eqref{eq:timeMomentum} by a forward Euler scheme yields
\begin{multline}
    \label{eq:time2}
    \frac{\mathbf{u}^{n+1} - \mathbf{u}^{n}}{\Delta t} + \text{convective}^{n} + \mathbb{E}^{(1,0)} P^{n+1} + \frac{1}{\text{Re}} \mathbb{H}^{(1,\tilde{1})} \tilde{\mathbb{E}}^{(1,0)} \mathbb{H}^{(\tilde{0},2)} \mathbb{E}^{(2,1)} \mathbf{u}^{n} \\
    + \frac{1}{\text{Re}} \mathbf{u}_{\text{pres}} = 0
\end{multline}
Why does $P$ have $n + 1$ in its superscript and not just $n$? The pressure at $t = 0$ is not required because $P^0$ does not appear in the equations. This means that $P$ must be associated with the next timestep. Multiplication of Equation \eqref{eq:time2} by $\Delta t$ gives
\begin{multline}
    \label{eq:time3}
    \mathbf{u}^{n+1} - \mathbf{u}^{n} + \Delta t \biggl( \text{convective}^{n} + \mathbb{E}^{(1,0)} P^{n+1} + \frac{1}{\text{Re}} \mathbb{H}^{(1,\tilde{1})} \tilde{\mathbb{E}}^{(1,0)} \mathbb{H}^{(\tilde{0},2)} \mathbb{E}^{(2,1)} \mathbf{u}^{n} \\
    + \frac{1}{\text{Re}} \mathbf{u}_{\text{pres}} \biggr) = 0
\end{multline}
Multiplying all terms in Equation \eqref{eq:time3} by $\tilde{\mathbb{E}}^{(2,1)} \mathbb{H}^{(\tilde{1},1)}$, we have
\begin{multline}
    \label{eq:time4}
    \tilde{\mathbb{E}}^{(2,1)} \mathbb{H}^{(\tilde{1},1)} \mathbf{u}^{n+1} - \tilde{\mathbb{E}}^{(2,1)} \mathbb{H}^{(\tilde{1},1)} \mathbf{u}^{n} + \tilde{\mathbb{E}}^{(2,1)} \mathbb{H}^{(\tilde{1},1)} \Delta t \biggl( \text{convective}^{n} + \mathbb{E}^{(1,0)} P^{n+1} \\
    + \frac{1}{\text{Re}} \mathbb{H}^{(1,\tilde{1})} \tilde{\mathbb{E}}^{(1,0)} \mathbb{H}^{(\tilde{0},2)} \mathbb{E}^{(2,1)} \mathbf{u}^{n} + \frac{1}{\text{Re}} \mathbf{u}_{\text{pres}} \biggr) = 0
\end{multline}
Equating Equation \eqref{eq:time4} to the continuity equation yields
\begin{multline}
    \label{eq:time5}
    \tilde{\mathbb{E}}^{(2,1)} \mathbb{H}^{(\tilde{1},1)} \mathbf{u}^{n+1} - \tilde{\mathbb{E}}^{(2,1)} \mathbb{H}^{(\tilde{1},1)} \mathbf{u}^{n} +  \tilde{\mathbb{E}}^{(2,1)} \mathbb{H}^{(\tilde{1},1)} \Delta t \biggl( \text{convective}^{n} + \mathbb{E}^{(1,0)} P^{n+1} \\
    + \frac{1}{\text{Re}} \mathbb{H}^{(1,\tilde{1})} \tilde{\mathbb{E}}^{(1,0)} \mathbb{H}^{(\tilde{0},2)} \mathbb{E}^{(2,1)} \mathbf{u}^{n} + \frac{1}{\text{Re}} \mathbf{u}_{\text{pres}} \biggr) = \tilde{\mathbb{E}}^{(2,1)} \mathbb{H}^{(\tilde{1},1)} \mathbf{u}^{n+1} + \mathbf{u}^{}_{\text{norm}}
\end{multline}
(\textit{Note:} It is perfectly acceptable to write the continuity equation as $\tilde{\mathbb{E}}^{(2,1)} \mathbb{H}^{(\tilde{1},1)} \mathbf{u}^{n+1}$. The superscript of $\mathbf{u}$ in the continuity equation can be chosen freely because the continuity equation holds true at \emph{any} given time.) The $\tilde{\mathbb{E}}^{(2,1)} \mathbb{H}^{(\tilde{1},1)} \mathbf{u}^{n+1}$ term can now be removed from both sides of Equation \eqref{eq:time5}. Doing so gives
\begin{multline}
    \label{eq:time6}
    - \tilde{\mathbb{E}}^{(2,1)} \mathbb{H}^{(\tilde{1},1)} \mathbf{u}^{n} +
    \tilde{\mathbb{E}}^{(2,1)} \mathbb{H}^{(\tilde{1},1)} \Delta t \biggl( \text{convective}^{n} + \mathbb{E}^{(1,0)} P^{n+1} \\
    + \frac{1}{\text{Re}} \mathbb{H}^{(1,\tilde{1})} \tilde{\mathbb{E}}^{(1,0)} \mathbb{H}^{(\tilde{0},2)} \mathbb{E}^{(2,1)} \mathbf{u}^{n} + \frac{1}{\text{Re}} \mathbf{u}_{\text{pres}} \biggr) = \mathbf{u}^{}_{\text{norm}}
\end{multline}
After some simple algebraic rearragenement of Equation \eqref{eq:time6}, we obtain
\begin{multline}
    \label{eq:time7}
    \tilde{\mathbb{E}}^{(2,1)} \mathbb{H}^{(\tilde{1},1)} \mathbb{E}^{(1,0)} P^{n+1} = \tilde{\mathbb{E}}^{(2,1)} \mathbb{H}^{(\tilde{1},1)} \biggl( \frac{\mathbf{u}^{n}}{\Delta t} - \text{convective}^{n} \\
    - \frac{1}{\text{Re}} \mathbb{H}^{(1,\tilde{1})} \tilde{\mathbb{E}}^{(1,0)} \mathbb{H}^{(\tilde{0},2)} \mathbb{E}^{(2,1)} \mathbf{u}^{n} - \frac{1}{\text{Re}} \mathbf{u}_{\text{pres}} \biggr) + \frac{\mathbf{u}^{}_{\text{norm}}}{\Delta t}
\end{multline}
which is equivalant to
\begin{equation}
    \label{eq:time8}
    A \mathbf{P}^{n+1} = f
\end{equation}
where
\begin{equation}
    \label{eq:time9}
    A = \tilde{\mathbb{E}}^{(2,1)} \mathbb{H}^{(\tilde{1},1)} \mathbb{E}^{(1,0)}
\end{equation}
and
\begin{multline}
    \label{eq:time10}
    f = \tilde{\mathbb{E}}^{(2,1)} \mathbb{H}^{(\tilde{1},1)} \biggl( \frac{\mathbf{u}^{n}}{\Delta t} - \text{convective}^{n} - \frac{1}{\text{Re}} \mathbb{H}^{(1,\tilde{1})} \tilde{\mathbb{E}}^{(1,0)} \mathbb{H}^{(\tilde{0},2)} \mathbb{E}^{(2,1)} \mathbf{u}^{n} \\
    - \frac{1}{\text{Re}} \mathbf{u}_{\text{pres}} \biggr) + \frac{\mathbf{u}^{}_{\text{norm}}}{\Delta t}
\end{multline}

(\textit{Note:} $\mathbf{P}^{n+1}$ in Equation \eqref{eq:time8} is written in boldface  because the numerical values of $P$ are stored in an array (i.e., a vector). However, the physical quantity $P$ remains a scalar quantity and is \emph{not} a vector quantity.) Once the system $A \mathbf{P}^{n+1} = f$ has been solved, $P^{n+1}$ can be substituted into
\begin{multline}
    \label{eq:time11}
    \mathbf{u}^{n+1} = \mathbf{u}^{n} - \Delta t \biggl( \text{convective}^{n} + \mathbb{E}^{(1,0)} P^{n+1} + \frac{1}{\text{Re}} \mathbb{H}^{(1,\tilde{1})} \tilde{\mathbb{E}}^{(1,0)} \mathbb{H}^{(\tilde{0},2)} \mathbb{E}^{(2,1)} \mathbf{u}^{n} \\
    + \frac{1}{\text{Re}} \mathbf{u}_{\text{pres}} \biggr)
\end{multline}
to compute the solution at time $n + 1$. Note that Equation \eqref{eq:time11} is the result of a simple algebraic rearrangement of Equation \eqref{eq:time3}.