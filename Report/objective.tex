Most of the established numerical methods for discretizing and solving partial differential equations (for instance, the finite-difference method, the finite-volume method, and the finite-element method) are based on the idea of approximating the derivative terms as closely as possible. This is a valid approach, but it has a number of disadvantages; numerical errors are introduced right from the start and the magnitude of the numerical error is a  function of the density of the mesh.  

The objective of this report is to present a novel method for solving the incompressible Navier-Stokes equations using tools from discrete exterior calculus (DEC). The method takes a geometric approach in order to preserve the physical relations rather than approximating them. A key ingredient of the idea is to avoid approximating derivatives entirely by expressing the variables as integrated quantities. The method is therefore mathematically exact until the last stages of the process. The method will be used to solve the incompressible Navier-Stokes equations on a two-dimensional lid-driven cavity because the lid-driven cavity flow problem is a classical test problem for the validation of Navier-Stokes codes. The main aim of this report is to demonstrate that the method works and to show its advantages over classical numerical methods.

In Chapter \ref{cha:physics} we will derive the governing equations and formally present the lid-driven cavity flow problem. In Chapter \ref{cha:mathematics} we will introduce the numerical framework in a step-by-step manner. In Chapter \ref{cha:code} we will show the most important excerpt of the program code and look for ways to optimize the code in terms of memory use and execution speed. Finally, in Chapter \ref{cha:results} we will validate the results using a 1997 paper by O.~Botella and R.~Peyret \parencite{botella1998benchmark}.