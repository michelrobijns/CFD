Many of the established numerical methods for discretizing and solving partial differential equations (for example, the finite-difference method, the finite-volume method, and the finite-element method) are based on the belief that the derivatives must be approximated as closely as possible. You will typically start by taking the governing equations and replacing the derivatives with some numerical approximation of the derivative. This is a valid approach, but it has a number of disadvantages; numerical errors are introduced right from the start and the magnitude of the numerical error is a  function of the density of the mesh.  

The objective of this report is to present a novel method for solving the incompressible Navier-Stokes equations using tools from discrete exterior calculus (DEC). The method takes a geometric approach in order to preserve the physical relations instead of approximating them. Part of the idea is to avoid approximating derivatives by expressing variables as integrals. The method is therefore mathematically exact up until one of the last stages of the process. The method will be used to solve the incompressible Navier-Stokes equations on a two-dimensional lid-driven cavity because the lid-driven cavity flow problem is a classical test problem for the validation of Navier-Stokes codes. The main aim of this report is to show that the method works and to indicate its advantages. There will be no comparisons with classic numerical methods in terms of accuracy or execution speed.

In Chapter \ref{cha:physics} we will derive the governing equations and formally present the lid-driven cavity flow problem. In Chapter \ref{cha:mathematics} we will introduce the numerical framework in a step-by-step manner. In Chapter \ref{cha:code} we will show the most important excerpt of the program code and look for ways to make the code more efficient in terms of memory use and execution time. Finally, in Chapter \ref{cha:results} we will validate the results using a 1997 paper by O.~Botella and R.~Peyret \parencite{botella1998benchmark}.