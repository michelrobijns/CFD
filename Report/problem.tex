\section{The Lid-Driven Cavity Flow Problem}

The incompressible dimensionless Navier-Stokes equations will be solved on the two-dimensional square domain $\Omega \equiv \left[ 0, 1 \right] \times \left[ 0, 1 \right]$, shown in Figure \ref{fig:problem}. The boundary of $\Omega$ is defined as $\partial \Omega \equiv \Gamma_1 \cup \Gamma_2 \cup \Gamma_3 \cup \Gamma_4$. 

The top boundary or "lid" of $\Omega$, $\Gamma_3$, imposes a shear stress on the fluid by moving leftward with a velocity of unity, hence the name lid-driven cavity flow problem. Thus, the boundary conditions are $\left. \mathbf{u} \right|_{\Gamma_3} = (-1, 0)$ and $\left. \mathbf{u} \right|_{\Gamma_1} = \left. \mathbf{u} \right|_{\Gamma_2} = \left. \mathbf{u} \right|_{\Gamma_4} = (0, 0)$. At $t = 0$, the fluid is at rest. That is, $ \left. \mathbf{u} \right|_{t = 0} \equiv 0$.

\begin{figure}[ht]
    \centering
    \begin{tikzpicture}
        % Draw lines
        \foreach \i in {0,6} {
            \draw (\i,0) -- (\i,6);
            \draw (0,\i) -- (6,\i);
        }
        % Draw x-axis and y-axis
        \draw [dashed, ->] (6,0) -- (7,0) node [right] {x};
        \draw [dashed, ->] (0,6) -- (0,7) node [above] {y};
        % Label corner points
        \node[below left=1ex] at (0,0) {$0$};
        \node[below=1ex] at (6,0) {$1$};
        \node[left=1ex] at (0,6) {$1$};
        % Label domain
        \node at (3,3) {$\Omega$};
        % Label boundaries
        \node[above=1ex] at (3,6) {$\Gamma_3$};
        \node[below=1ex] at (3,0) {$\Gamma_1$};
        \node[left=1ex] at (0,3) {$\Gamma_4$};
        \node[right=1ex] at (6,3) {$\Gamma_2$};
    \end{tikzpicture}
    \caption{The lid-driven cavity flow problem.}
    \label{fig:problem}
\end{figure}

% Checked 4/24/16 8:12 PM
